\documentclass[12pt, letterpaper]{article}


\usepackage[a4paper, margin=3cm, twoside]{geometry} % To create a small page
\usepackage{fancyhdr}
\usepackage[bottom]{footmisc} % always place footer at bottom of page 

\usepackage{graphicx} %LaTeX package to import graphics
\usepackage{marginnote}

\title{My first LaTeX document}
\author{Hubert Farnsworth}
\date{August 2022}

\renewcommand*\contentsname{Inhaltsverzeichnis} % Change the title of the Table of contents

%----------------------------------------------------------------------------------------
%   HEADER AND FOOTER CONFIGURATION
%----------------------------------------------------------------------------------------

\pagestyle{fancy}
\fancyhead{} % clear header
\fancyfoot{} % clear footer
\renewcommand{\headrulewidth}{0pt}
\fancyhead[R]{\rightmark}
\fancyfoot[LE,RO]{\thepage} % page at left on even and right on odd pages

\begin{document}

\maketitle
\thispagestyle{empty}
\newpage

\tableofcontents
\thispagestyle{empty}
\newpage

\setcounter{page}{1}

\section{Features}

\subsection{Bold/...}

First document.\footnote{Or at least it was back then}
\paragraph{bold}
Some of the \textbf{greatest}
\paragraph{underline}
discoveries in \underline{science} 
\paragraph{kursiv}
were made by \textbf{\textit{accident}}.

\subsection{Images}

\begin{figure}[h]
    \centering
    \includegraphics[width=0.25\textwidth]{image}
    \caption{A nice plot.}
    \label{fig:mesh1}
\end{figure}

\subsubsection{Referencing Figures}

As you can see in figure \ref{fig:mesh1},
\marginnote{This is a margin note shifted 2cm, \textit{up} the page, relative to the line in which it is typeset.}[-2cm]
the function grows near the origin. 
This example is on page \pageref{fig:mesh1}.
\reversemarginpar\marginnote{This is another margin note but shifted 4cm \textit{up} the page, relative to the line in which it is typeset. It is also in the left-hand margin.}[-4cm]



\subsection{Lists}

\begin{itemize}
    \item The individual entries are indicated with a black dot, a so-called bullet.
    \item The text in the entries may be of any length.
\end{itemize}
  
\begin{enumerate}
    \item This is the first entry in our list.
    \item The list numbers increase with each entry we add.
\end{enumerate}

\subsection{Math}

This is a simple math expression \(\sqrt{x^2+1}\) inside text. 
And this is also the same: 
\begin{math}
\sqrt{x^2+1}
\end{math}
but by using another command.

This is a simple math expression without numbering
\[\sqrt{x^2+1}\] 
separated from text.

This is also the same:
\begin{displaymath}
\sqrt{x^2+1}
\end{displaymath}

And this is a numberd math: 
\begin{equation}
E=m
\end{equation}

\end{document}